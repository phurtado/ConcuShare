\documentclass[a4paper,10pt]{article}
\usepackage[spanish]{babel} % Paquetes de idioma
\usepackage[latin1]{inputenc} % Paquetes de idioma
\usepackage{graphicx} % Paquete para ingresar gráficos
\usepackage{grffile}
\usepackage{hyperref}
\usepackage{fancybox}
\usepackage{amsmath}
\usepackage{listings}

% Encabezado y Pié de página
\input{EncabezadoyPie.tex}
% Carátula del Trabajo
\title{ \input{Portada.tex} }

\begin{document}
	\maketitle % Hace que el título anterior sea el principal del documento
	\newpage

	\tableofcontents % Esta línea genera un indice a partir de las secciones y subsecciones creadas en el documento
	\newpage

	\section{Introducci\'on}
		El presente trabajo consiste en el desarrollo de una aplicaci\'on {\it concurrente} conocida como \emph{ConcuShare}. El objetivo de la misma
		es permitir el intercambio de archivos entre distintos usuarios. \\
		\indent Debido a que la materia apunta a obtener conocimientos sobre los mecanismos de concurrencia vistos hasta el momento en la c\'atedra, 
    el proyecto debe correr en una \'unica computadora y funcionar bajo un ambiente \emph{Unix/Linux} dado que los mecanismos vistos son los 
    implementados por \emph{System V}. \\
		\indent La implementaci\'on de la aplicaci\'on consiste en un esquema \emph{Cliente-Servidor}. El servidor se encuentra escuchando peticiones
		del cliente. Los clientes deben conectarse al servidor, y luego de esto pueden utilizar los servicios provistos por el mismo.
		\vspace{0.5cm}

	\section{Modo de Operaci\'on}

		\subsection {C\'omo compilar y correr la aplicaci\'on}
			Para poder correr la aplicaci\'on correctamente, lo primero que debe realizarse es compilar el programa. Para esto se ha provisto de un
			\emph{Makefile} que se encagar\'a de realizar el proceso de compilaci\'on. \\
			\indent Para compilar la aplicaci\'on servidor se debe ingresar el siguiente comando:
			\begin{verbatim}
				make server
			\end{verbatim}
			\indent Para compilar la aplicaci\'on cliente se debe ingresar el siguiente comando:
			\begin{verbatim}
				make client
			\end{verbatim}
			\indent Para compilar la aplicaci\'on que se encarga de realizar las transferencias, se debe ingresar el siguiente comando:
			\begin{verbatim}
				make transf
			\end{verbatim}
			\indent En el caso de que desee compilar todas las aplicaciones juntas puede hacerlo por medio de alguno de los siguientes comandos:
			\begin{verbatim}
				make server client transf
				make all
			\end{verbatim}

		\subsection{Casos de Uso}

			Como bien se dijo, la aplicaci\'on est\'a compuesta por un \emph{Servidor} que escucha peticiones y \emph{Clientes} que realizan las mismas. 
			De esta forma, el \emph{Servidor} es un proceso que no tiene interacci\'on con el usuario. En cambio, el \emph{Cliente} si tiene interacci\'on
			con el usuario. La aplicaci\'on \emph{Cliente} despliega un men\'u con las tareas que puede realizar el mismo. A continuaci\'on se 
			explica el modo de uso de cada una de las opciones del men\'u: 

			\begin{itemize}
				\item \textbf{1-Conectarse al Servidor:} Lo primero que debe hacer el cliente al iniciar la aplicaci\'on es conectarse a la misma. En el caso
				de no conectarse no podr\'a realizar ninguna otra tarea.
				\item \textbf{2-Desconectarse del Servidor:} Si el cliente desea dejar la aplicaci\'on debe desconectarse del servidor. Al desconectarse, el
				servidor remueve de la lista de archivos compartidos todos los ficheros compartidos por este usuario.
				\item \textbf{3-Compartir Archivos:} Mediante esta opci\'on, el cliente le indica al servidor que archivos quiere disponer para ser compartidos.
				El mismo debe pasarle la ruta del archivo a compartir. La misma debe ser v\'alida.
				\item \textbf{4-Descompartir Archivos:} Mediante esta opci\'on, el cliente le indica al servidor que archivos desea dejar de compatir con los 
				dem\'as usuarios. El mismo debe pasarle la ruta del archivo a descompartir, la cual debe ser v\'alida.
				\item \textbf{5-Descargar Archivos:} Mediante esta opci\'on, el usuario puede elegir que partidos compartidos desea descargarse. Debe ingresar
				para esto el pid del cliente que se encuentra compartiendo el archivo, el path del archivo compartido y el path de destino.
				\item \textbf{6-Obtener Lista de Archivos Compartidos:} Mediante esta opci\'on, el cliente obtiene la lista de los archivos compartidos por 
				todos los clientes que se encuentran conectados. La lista muestra los archivos compartidos por cada cliente, de modo que teniendo esta 
				informaci\'on se puede realizar la \emph{Descarga} de alg\'un archivo.
				\item \textbf{7-Salir de la Aplicaci\'on:} Mediante esta opci\'on, el cliente abandona la aplicaci\'on. \emph{Se deber\'ia desconectar el cliente
				antes de poder salir???}
			\end{itemize}

	\section{An\'alisis de la Soluci\'on}
		\subsection{Divisi\'on de la Aplicaci\'on en Procesos}
			Debido a la naturaleza de la soluci\'on adoptada, se pueden distinguir f\'acilmente los siguientes procesos:

			\begin{itemize}
				\item \textbf{Servidor:} Este proceso se encarga de escuchar las peticiones los clientes. Mientras ning\'un usuario le env\'ie ning\'un 
				mensaje, el mismo se bloquea. Cuando un cliente le env\'ia alg\'un mensaje, despierta al mismo y \'este se encarga de responderle al 
				usuario.
				\item \textbf{Cliente:} Al abrir un usuario una aplicaci\'on Cliente, se crea un nuevo proceso el cual puede comunicarse con el servidor.
				El Cliente env\'ia peticiones al servidor esperando una respuesta del mismo.
				\item \textbf{Transferidor:} Este proceso se crea cada vez que un cliente desea descargarse un archivo compartido. Tanto el Servidor como
				el Cliente crean un proceso \emph{Transferidor} por el cual se comunican para realizar la transferencia.
			\end{itemize}

		\subsection{Comunicaci\'on entre Procesos}
\end{document}
